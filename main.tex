\documentclass[dvipdfmx]{thesis}
\begin{document}
\makethesistitle[
type={卒業論文},
organization={砂大学},
division={砂部砂学科},
title={論文のタイトル\\サブタイトル},
studentnumber={B99T9999Z},
author={鳥取 太郎},
date={令和6年2月},
]
% \makethesistitle[
% type={修士論文},
% organization={砂大学大学院},
% division={持続性砂科学研究科砂学専攻},
% title={論文のタイトル\\サブタイトル},
% studentnumber={M99T9999Z},
% author={鳥取 太郎},
% date={令和6年2月},
% ]
\maketableofcontents
%%%%%%%%%%%%%%%%%%%%%%%%%%%%%%%%%%%%%%%%
\chapter{はじめに}
これは卒論・修論の本論用 \LaTeX テンプレートです.公式ではありません.
%%%%%%%%%%%%%%%%%%%%%%%%%%%%%%%%%%%%%%%%
\chapter{関連研究}
参考文献の参照例を示す\cite{Article:2024:Sunadai}.
%%%%%%%%%%%%%%%%%%%%%%%%%%%%%%%%%%%%%%%%
\chapter{提案手法}
%%%%%%%%%%%%%%%%%%%%%%%%%%%%%%%%%%%%%%%%
\section{図}
\figref{fig:example-image-1}に例を示す.

\begin{figure}[h]
  \centering
  \includegraphics[width=\columnwidth]{example-image}
  \caption{図の例}
  \label{fig:example-image-1}
\end{figure}

%%%%%%%%%%%%%%%%%%%%%%%%%%%%%%%%%%%%%%%%
\section{表}
\tabref{tab:example-example}に例を示す.

\begin{table}[h]
  \centering
  \caption{表の例}
  \begin{tabular}{|c c c|}
    a & b & c \\
    \hline
    00 & 01 & 02 \\ 
    10 & 11 & 12 \\ 
    20 & 21 & 22 \\ 
  \end{tabular}
  \label{tab:example-example}
\end{table}
%%%%%%%%%%%%%%%%%%%%%%%%%%%%%%%%%%%%%%%%
\chapter{実験}
%%%%%%%%%%%%%%%%%%%%%%%%%%%%%%%%%%%%%%%%
\chapter{考察}
%%%%%%%%%%%%%%%%%%%%%%%%%%%%%%%%%%%%%%%%
\chapter{おわりに}
%%%%%%%%%%%%%%%%%%%%%%%%%%%%%%%%%%%%%%%%
\begin{acknowledge}
謝辞をここに記述します.
\begin{flushright}
令和6年2月 鳥取 太郎
\end{flushright}
\end{acknowledge}
%%%%%%%%%%%%%%%%%%%%%%%%%%%%%%%%%%%%%%%%
\bibliographystyle{junsrt}
\bibliography{bib-paper,bib-book,bib-misc}
%%%%%%%%%%%%%%%%%%%%%%%%%%%%%%%%%%%%%%%%
\end{document}
